\documentclass{muratcan_cv}

\setname{Alexandre Lachance}{}
\setaddress{Québec, Québec, Canada}
\setmobile{+1 514-550-6196}
\setmail{alexandrelachance@me.com}
\setthemecolor{Black}
\setlinkedinaccount{https://www.linkedin.com/in/alexandre-lachance-36b497164/} 
\setgithubaccount{https://github.com/AlexandreLachanceGit} 

\begin{document}

%Créer en-tête
\headerview
\vspace{1ex} % espace blanc
%
\section{Éducation}
\datedexperience{Maîtrise en Génie Logiciel (M.Sc.A.)}{Janv 2023 – Déc 2024}
\explanation{Université McMaster | Date prévue : Octobre 2024 | Moyenne : 3,98/4,00}{Hamilton, ON}
\datedexperience{Baccalauréat en Informatique et Génie Logiciel (Honor)}{Sept 2020 – Déc 2022}
\explanation{Université du Québec à Montréal | Moyenne : 3,90/4,30}{Montréal, QC}
\datedexperience{Diplôme d'études collégiales (DEC) en Informatique et Mathématiques}{Sept 2018 – Mai 2020}
\explanation{Cégep Lionel-Groulx}{Sainte-Thérèse, QC}

\section{Compétences \& Intérêts}
\smallskip
\newcommand{\skillone}{\createskill{Langages de Programmation}{Rust \cpshalf Python \cpshalf Java \cpshalf C \cpshalf TypeScript \cpshalf C\#}}
%
\newcommand{\skilltwo}{\createskill{Technologies}{Linux \cpshalf Git \cpshalf GitHub CI/CD \cpshalf Docker \cpshalf AWS \cpshalf MPS \cpshalf Langium}}
%
\newcommand{\skillthree}{\createskill{Intérêts}{Architecture Logicielle \cpshalf Langages de Programmation \cpshalf Programmation Générative}}
%
\newcommand{\skillfour}{\createskill{Langues}{Français \cpshalf Anglais}}
%
\createskills{\skillfour, \skillone, \skilltwo, \skillthree}
\vspace{-3mm} % réduire l'espace blanc !

\section{Expériences \& Projets}
\datedexperience{Thèse de Maîtrise | Développement de Lever : Framework pour Éditeurs de DSLs}{Janv 2023 –  Déc 2024}
\explanation{Université McMaster | Supervision de \href{https://mosser.github.io/}{Sébastien Mosser}}{Hamilton, ON}
\explanationdetail{
	\smallskip
	\coloredbullet\ %
	Développement d’un \textit{framework} léger et simple d'utilisation servant à la génération de serveurs de langage dédiés aux DSLs (Domain-Specific Languages).

	\smallskip
	\coloredbullet\ %
	Utilisation de Rust et de ses capacités de génération de code pour un développement efficace.

	\smallskip
	\coloredbullet\ %
	Développement d'un serveur de langage et d'une extension Visual Studio Code en utilisant Lever pour P4, le langage standard pour programmer les réseaux définis par logiciel (SDN).

	\smallskip
	\coloredbullet\ %
	Écriture de la grammaire Tree-sitter (bibliothèque d'analyseur syntaxique) pour P4 et contribution à Tree-sitter.

	\smallskip
	\coloredbullet\ %
	Création de langages de programmation en utilisant les outils Langium et MPS.

	\smallskip
	\coloredbullet\ %
	Collaboration avec des partenaires industriels (Intel, Telus) pour valider l'implémentation du support P4.
}
%
\datedexperience{Assistant d'Enseignement | Programmation \& Conception Logicielle}{Mai 2023 – Juin 2024}
\explanation{Université McMaster}{Hamilton, ON}
\explanationdetail{
	\smallskip
	\coloredbullet\ %
	Aidé les étudiants à comprendre les concepts clés de programmation (récursivité, flux de contrôle, etc.).

	\smallskip
	\coloredbullet\ %
	Aidé les étudiants avec les fondations de la conception logicielle (modèles de conception, exigences).

	\smallskip
	\coloredbullet\ %
	Conduite de laboratoires, correction des devoirs, et soutien individuel pendant les heures de bureau.
}
%
\datedexperience{Assistant de Recherche | Développement de Plateforme de Recherche Full-Stack}{Janv 2022 – Janv 2023}
\explanation{Université du Québec à Montréal | Supervision de \href{https://mosser.github.io/}{Sébastien Mosser}}{Montréal, QC}
\explanationdetail{
	\smallskip
	\coloredbullet\ %
	Développement d'une application web avec un frontend en Angular (TypeScript) et un backend en Flask (Python).

	\smallskip
	\coloredbullet\ %
	Permis aux chercheurs de collecter et d'analyser des ensembles de données massifs de Tweets.

	\smallskip
	\coloredbullet\ %
	Implémentation de la comparaison d'ensembles de données pour identifier des tendances comportementales.

	\smallskip
	\coloredbullet\ %
	Collaboration avec des professeurs en communication pour adapter la plateforme à leurs besoins.
}
%
\datedexperience{Stagiaire de Recherche | IA pour la Simplification Automatique de Textes}{Juin 2022 – Août 2022}
\explanation{Université du Québec à Montréal | Supervision de \href{https://professeurs.uqam.ca/professeur/meurs.marie-jean/}{Marie-Jean Meurs}}{Montréal, QC}
\explanationdetail{
	\smallskip
	\coloredbullet\ %
	Recherche de solutions basées sur l'IA pour la simplification de textes en explorant l'état de l'art.

	\smallskip
	\coloredbullet\ %
	Maîtrise des techniques de traitement du langage naturel (NLP) et d'apprentissage automatique (ML).

	\smallskip
	\coloredbullet\ %
	Test, débogage et évaluation de modèles existants pour la simplification de textes.

	\smallskip
	\coloredbullet\ %
	Collaboration étroite avec Radio-Canada / CBC pour des applications pratiques de simplification de textes.
}
%
\datedexperience{Instructeur de Programme / Tuteur}{Janv 2022 – Avr 2022}
\explanation{Université du Québec à Montréal}{Montréal, QC}
\explanationdetail{
	\smallskip
	\coloredbullet\ %
	Aide aux étudiants de premier cycle avec le matériel de cours, les devoirs et autres projets.

	\smallskip
	\coloredbullet\ %
	Coordination avec les assistants d'enseignement pour aider les étudiants en difficulté.
	\smallskip
}
%
\datedexperience{Projet Final de DEC | Jeu en Réalité Virtuelle}{Sept 2020 – Déc 2020}
\explanation{Cégep Lionel-Groulx}{Sainte-Thérèse, QC}
\explanationdetail{
	\smallskip
	\coloredbullet\ %
	Développement d'un jeu en réalité virtuelle dans Unity (C\#), avec un accent sur l'immersion et l'interaction.

	\smallskip
	\coloredbullet\ %
	Implémentation de cartes générées procéduralement et de physiques de combat.
}
\pagebreak
%
\section{Publications}
%
\publication{Building Deduplicated Model Repositories to Assess Domain-Specific Languages Evolution}{Sept 2024}{\underline{Alexandre Lachance}, Sébastien Mosser}{\href{https://doi.org/10.1145/3652620.3688338}{10.1145/3652620.3688338}}
\explanationdetail{
	\smallskip
	\coloredbullet\ %
	Développement d'un pipeline de déduplication pour les répertoires de modèles.

	\smallskip
	\coloredbullet\ %
	Application du pipeline pour créer un ensemble de données et identifier des doublons sur plusieurs forges.

	\smallskip
	\coloredbullet\ %
	Validation du pipeline en validant l'évolution d'outils avec l'ensemble de données obtenu.

	\smallskip
	\coloredbullet\ %
	Démonstration de l'efficacité de l'approche pour réduire la complexité computationnelle de la déduplication.
}
%
\publication{Automatic Text Simplification of News Articles in the Context of Public Broadcasting}{Déc 2022}{Diego Maupomé, Fanny Rancourt, Thomas Soulas, \underline{Alexandre Lachance}, \emph{et al.}}{\href{https://arxiv.org/abs/2212.13317}{10.48550/arXiv.2212.13317}}
\explanationdetail{
	\smallskip
	\coloredbullet\ %
	Simplification automatique de textes pour améliorer l'accessibilité des articles de CBC/Radio-Canada.

	\smallskip
	\coloredbullet\ %
	Comparaison des systèmes lexicaux avec des méthodes d'apprentissage profond pour la simplification.

	\smallskip
	\coloredbullet\ %
	Évaluation des performances avec des méthodes manuelles et automatiques.

	\smallskip
	\coloredbullet\ %
	Résultats prometteurs obtenus avec des modèles basés sur la paraphrase pour la simplification de phrases.
}
\vspace{0.5em}
%
\section{Présentations}
%
\datedexperience{Building Deduplicated Model Repositories to Assess Domain-Specific Languages Evolution}{Sept 2024}
\explanation{Models and Evolution Workshop, Models 2024}{Linz, Autriche}
%
\datedexperience{From Zero to VS Code: A Framework Approach to Language Support}{Nov 2023}
\explanation{MDENet Research Demonstration | Lien vers la vidéo : \scriptsize \url{https://www.youtube.com/watch?v=JzCYxz4G_Cc}}{En ligne}
%
\datedexperience{Developing a Modular Language Server to Support P4 Developers}{Sept 2023}
\explanation{P4 Developers Days Meeting}{En ligne}
%
\datedexperience{A Language Engineering Approach to Support the P4 Coding Ecosystem}{Avr 2023}
\explanation{P4 Workshop at Intel HQ | Lien vers la publication : \scriptsize \url{https://macsphere.mcmaster.ca/handle/11375/28616}}{San Francisco, USA}
%
\section{Activités Extrascolaires}
\datedexperience{Secrétaire Exécutif | Association Étudiante}{Juin 2022 – Déc 2022}
\explanation{AGEEI (Association Générale des Étudiants en Informatique) à l'Université du Québec à Montréal}{Montréal, QC}
\explanationdetail{
	\smallskip
	\coloredbullet\ %
	Participation aux réunions exécutives et générales, rédaction des procès-verbaux.

	\smallskip
	\coloredbullet\ %
	Gestion de la correspondance officielle et archivage des documents légaux/administratifs.

	\smallskip
	\coloredbullet\ %
	Veille au respect des décisions de l'Assemblée Générale.

	\smallskip
	\coloredbullet\ %
	Gestion d'une communauté en ligne de plus de 1200 étudiants, en répondant aux questions liées aux cours.
}
\datedexperience{Membre de l'Équipe de Compétition en Cybersécurité}{Sept 2021 – Déc 2022}
\explanation{Université du Québec à Montréal}{Montréal, QC}
\explanationdetail{
	\smallskip
	\coloredbullet\ %
	Participation à des événements "Capture The Flag" tels que Hackfest 2021, NorthSec 2022 et Hackfest 2022.

	\smallskip
	\coloredbullet\ %
	Collaboration avec des pairs pour résoudre des défis et approfondir les connaissances en cybersécurité.
}

\smallskip

\section{Autres Réalisations}
\indent
\explanationdetail{
	\smallskip
	\coloredbullet\ %
	\textbf{Bourse d'Excellence} accordée par le gouvernement du Québec en 2021.
}
\end{document}
